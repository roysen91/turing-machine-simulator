\documentclass{beamer}
\usepackage{tikz}
\usepackage{animate}
\usepackage{fp}
\def \step{0.5}
% define the bounding box
\def \boundb{(-4,-2.5) rectangle (4,2.5)}
\def \band{[step=0.5cm](-4,-0.5) grid (4,0)}
\def \triangle{(0,0.25) -- (0.5,0.25) -- (0.25,0) -- cycle}
\def \state{(0,0.25) rectangle (0.5,0.75)}
\begin{document}
\begin{frame}
\frametitle{Turingmaschine}
\begin{center}
\begin{animateinline}[autoplay]{1}
\begin{tikzpicture} 
\draw \boundb; 
\draw \band; 
\draw \triangle;
\draw \state;
\node at (0.25,0.5) {{$q0$}};
\node at (-3,2) {Steps: {1}};
\def \start  {0.25 -2* \step}
\foreach \bit in {B,B,0,1,0,B,B}{
\node at (\start,-0.25) {\bit};
\FPeval{start}{\start+\step}
\xdef\start{\start}% make \xb global
% resets bounding box (crops right overflow)
\pgfresetboundingbox
\path[use as bounding box, draw] \boundb;
}
\end{tikzpicture}
\newframe
\begin{tikzpicture} 
\draw \boundb; 
\draw \band; 
\draw \triangle;
\draw \state;
\node at (0.25,0.5) {{$q0$}};
\node at (-3,2) {Steps: {2}};
\def \start  {0.25 -3* \step}
\foreach \bit in {B,B,B,1,0,B,B}{
\node at (\start,-0.25) {\bit};
\FPeval{start}{\start+\step}
\xdef\start{\start}% make \xb global
% resets bounding box (crops right overflow)
\pgfresetboundingbox
\path[use as bounding box, draw] \boundb;
}
\end{tikzpicture}
\newframe
\begin{tikzpicture} 
\draw \boundb; 
\draw \band; 
\draw \triangle;
\draw \state;
\node at (0.25,0.5) {{$q1$}};
\node at (-3,2) {Steps: {3}};
\def \start  {0.25 -4* \step}
\foreach \bit in {B,B,B,1,0,B,B}{
\node at (\start,-0.25) {\bit};
\FPeval{start}{\start+\step}
\xdef\start{\start}% make \xb global
% resets bounding box (crops right overflow)
\pgfresetboundingbox
\path[use as bounding box, draw] \boundb;
}
\end{tikzpicture}
\newframe
\begin{tikzpicture} 
\draw \boundb; 
\draw \band; 
\draw \triangle;
\draw \state;
\node at (0.25,0.5) {{$q1$}};
\node at (-3,2) {Steps: {4}};
\def \start  {0.25 -5* \step}
\foreach \bit in {B,B,B,1,0,B,B}{
\node at (\start,-0.25) {\bit};
\FPeval{start}{\start+\step}
\xdef\start{\start}% make \xb global
% resets bounding box (crops right overflow)
\pgfresetboundingbox
\path[use as bounding box, draw] \boundb;
}
\end{tikzpicture}
\newframe
\begin{tikzpicture} 
\draw \boundb; 
\draw \band; 
\draw \triangle;
\draw \state;
\node at (0.25,0.5) {{$q2$}};
\node at (-3,2) {Steps: {5}};
\def \start  {0.25 -4* \step}
\foreach \bit in {B,B,B,1,0,B,B}{
\node at (\start,-0.25) {\bit};
\FPeval{start}{\start+\step}
\xdef\start{\start}% make \xb global
% resets bounding box (crops right overflow)
\pgfresetboundingbox
\path[use as bounding box, draw] \boundb;
}
\end{tikzpicture}
\newframe
\begin{tikzpicture} 
\draw \boundb; 
\draw \band; 
\draw \triangle;
\draw \state;
\node at (0.25,0.5) {{$q2$}};
\node at (-3,2) {Steps: {6}};
\def \start  {0.25 -3* \step}
\foreach \bit in {B,B,B,1,B,B,B}{
\node at (\start,-0.25) {\bit};
\FPeval{start}{\start+\step}
\xdef\start{\start}% make \xb global
% resets bounding box (crops right overflow)
\pgfresetboundingbox
\path[use as bounding box, draw] \boundb;
}
\end{tikzpicture}
\newframe
\begin{tikzpicture} 
\draw \boundb; 
\draw \band; 
\draw \triangle;
\draw \state;
\node at (0.25,0.5) {{$q3$}};
\node at (-3,2) {Steps: {7}};
\def \start  {0.25 -2* \step}
\foreach \bit in {B,B,B,1,B,B,B}{
\node at (\start,-0.25) {\bit};
\FPeval{start}{\start+\step}
\xdef\start{\start}% make \xb global
% resets bounding box (crops right overflow)
\pgfresetboundingbox
\path[use as bounding box, draw] \boundb;
}
\end{tikzpicture}
\newframe
\begin{tikzpicture} 
\draw \boundb; 
\draw \band; 
\draw \triangle;
\draw \state;
\node at (0.25,0.5) {{$q17$}};
\node at (-3,2) {Steps: {8}};
\def \start  {0.25 -3* \step}
\foreach \bit in {B,B,B,1,B,B,B}{
\node at (\start,-0.25) {\bit};
\FPeval{start}{\start+\step}
\xdef\start{\start}% make \xb global
% resets bounding box (crops right overflow)
\pgfresetboundingbox
\path[use as bounding box, draw] \boundb;
}
\end{tikzpicture}
\newframe
\end{animateinline}
\end{center}
\end{frame}
\end{document}
